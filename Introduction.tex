\chapter{Introduction}


At its core quantum computation combines two of the biggest scientific and technological breakthroughs of the last century: the first being quantum mechanics with its strange and counter-intuitive interpretations, and the second being integrated logic circuits which gave rise to digital computers and fueled the information age. The central tenet of quantum information processing is to replace the classical bits that are either `1' OR `0', with quantum bits (``qubits") that can be in a coherent superposition of `1' AND `0'. However, the practical challenge of preserving these superpositions is daunting. This is because quantum systems typically couple to uncontrolled degrees of freedom in their environment that cause them to decay into classical mixtures \cite{Wiring_up_quantum_systems}. This poses the primary challenges for quantum engineers today -- namely decoherence, or the phenomenon that destroys quantum information due to parasitic coupling of qubits to uncontrolled environmental degrees of freedom. Traditionally the efforts to mitigate decoherence have focused on eliminated or minimizing such unwanted interactions. However, it has recently been shown that system-environment couplings can be ``engineered" in a manner that allows us to use dissipation to our advantage. In particular, the system can be driven into a pure state state by the dissipation, if the environment serves as a Maxwell demon and evacuates entropy from the quantum system of interest 
\cite{Dissipative_Universal_Quantum_Computing}. Such an approach has even been extended to stabilize multi-particle states known as matrix-product states. \cite{Dissipative_State_Engineering_Zoller, Cooling_Atoms_into_Entangled_States,PhysRevLett.106.090502}. 

Use of dissipation engineering to stabilize entangled states has recently gained a lot of theoretical and experimental attention. This is because entanglement is important to many quantum information applications. One example is quantum teleportation, a process by which quantum information can be transmitted from one location to another over a classical communication channel if the sender and receiver share an entangled pair of qubits. Entangled states also form the basis of quantum cryptography and quantum error correction protocols. However, such states are particularly sensitive to deocherence since local noise seen by any of the qubits forming the entangled state can destroy the quantum information encoded in non-local correlations. Thus preserving entanglement for long times is a simultaneously a challenging and compelling prospect. In this thesis, we specifically focus on dissipative stabilization of two- and four-qubit entangled states with high-fidelity.

This thesis is organized as follows: in chapter 2, we introduce open quantum system formalism and present the derivation of quantum master equation that captures the influence of the environment on a quantum system. Using the master equation to describe a single qubit coupled to a radiation field at finite temperature, we show how dissipation into the environment leads to decoherence. In chapter 3, we introduce the notion of dark states that are the pure steady states of an open system. We detail the method of adiabatic elimination which allows identification of dark state of an open system, and use it to study stabilization of a Bell state in a system of two qubits coupled to the fundamental mode of a resonator (described using  a simple harmonic oscillator). We compare our results for fidelity and rate of Bell state stabilization for the cases when flow of excitations in the two-qubit subsystem is bidirectional (symmetric) vs unidirectional (chiral). In chapter 4, we extend our scheme to four-qubits and show how purification of multipartite entanglement can be achieved in this framework. Chapter 5 presents a brief summary of our results and potential extensions for future studies.


%Furthermore, ``tools" that allow precise control over these systems that operate in quantum domain need to be developed as well as a way quantum computer the qubits need to be `wired-up' in such a way that they can interact and exchange quantum-information. Some may call this a ``quantum network" \cite{The_Quantum_Internet}.

%While natural candidates for qubits include atoms, ions, have the and spins artificial systems, such as super-conducting circuits and quantum dots, designed and fabricated to meet specification. 

%In replacing bits with qubits the first question that must be addressed is, can a circuit display quantum behavior. In the mid 1980s, the quantum behavior of a superconducting circuit, that incorporated a Josephson junction, was observed in a tunneling experiment of the zero-voltage state \cite{Macroscopic_Tunneling_of_Macroscopic_Variable}. Later energy quantization of superconducting circuits was demonstrated through microwave spectroscopy \cite{Quantum_Mechanics_Macroscopic}. By the end of the 1990s scientist had demonstrated that superconducting 'island' circuits, or charge qubit, could be in superposition of different charge states \cite{Circuit_Superposition}. Coherent control of state evolution for these charge superpositions has also been demonstrated \cite{Coherent_Control_of_Circuit}.

%The second question that needs to be addressed is how do we realize quantum communication between the qubits. Given that qubits interact strongly with the surrounding electromagnetic field, optical photons are a natural choose for communicate between ``quantum nodes" \cite{Qubit_EM_Coupling}. This approach is frequently called circuit quantum electrodynamics (circuit QED), and is similar to the branch of atomic physics that studies the interactions between atoms and photons at the single photon level (cavity QED). In circuit QED microwave photons are created from transmission line with gaps in the wires the are an integer number of a half wavelength apart. These photons are coupled to superconducting qubits that act as artificial atoms. The qubit coherently absorbs and re-emits the microwave photon many times before it decays into other modes; dubbed the strong-coupling regime (ie. $g \gg \kappa , \gamma$). The quality factor for these vaccumn Rabi oscillations can be as high as $10^6$ \cite{Wiring_up_quantum_systems}. These large quality factors have made coherent state transfer between qubits possible, and thus demonstrate qubits can be `wire-up' with photons. Quantum communication in superconducting qubits has been demonstrated in both charge and flux qubits. \cite{Charge_Qubit_Quantum_Bus,Flux_Qubit_Quantum_Bus}.