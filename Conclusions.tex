\chapter{Conclusion}

In conclusion, we have shown that even though dissipation can lead to decohernce that washes away quantum information, it possible to use dissipation to our advantage to stabilize entanglement with high-fidelity. Specifically, we presented schemes for two- and four-qubit entanglement generation, employing an ``engineered" qubit-reservoir coupling. The reservoir in question can be implemented via a lossy harmonic oscillator. Such a scheme is compatible with standard circuit-QED platforms using microwave resonators and superconducting qubits. 

We showed how the method of adiabatic elimination provides a convenient route to identify ``engineered" jump operator that captures the effective dissipation seen by the qubits. This allows us to study open system dynamics of only the reduced system of interest, giving considerable advantage in terms of both analytical tractability and numerical computation. We found that for parameters considered here, entanglement stabilization with chiral qubit interactions outperforms the scheme that relies on symmetric qubit interactions. This was quantified in terms of a new performance metric $M$: realizing a large value of this metric entails a \emph{simultaneous} maximization of fidelity and rate of stabilization. The chiral scheme had a performance metric that was 10x bigger than the symmetric scheme for optimal parameters. 

Furthermore, four-qubit studies show that chirality is essential to purify mixtures into genuine multipartite entangled states. The `knob' that determines the nature of entanglement in a 1D chiral qubit chain is the detuning pattern of the driving field: if the detunings are ``alternating" the qubits stabilize into pairs of singlets, while when the detunings are ``staggered" (realized by permuting one of the detunings), the steady state is a true tetramer. Multipartite entanglement stabilization may be possible from symmetric dissipative interactions alone, but this requires specific driving conditions along with initial state preparation. We also find that the time scales and dynamics in qubits chains are qualitatively different with and with out chirality, which will form the basis of future investigations.

%In this thesis we have have considered the dynamics of dissipative engineering protocols and shown that multi-qubit entanglement is possible when coupled to a low Q-value cavity mode. These schemes always requires a strong resonate drive and display a trade off between fidelity and gap. However, the addition of chiral coupling seems to make the overall stabilization scheme better. Furthermore, chiral couplings are necessary for the purification of a four way entangled. 

%Finally, I would like to discuss future directions that this research can go. Also like to discuss how our work fits in with the broader field of chiral quantum optics. While in this thesis we considered qubits coupled to a mutual cavity mode one could consider qubits/spins coupled to a chiral wave-guide (see Appendix). Spins coupled to a chiral wave-guide have applications quantum communication protocols. One possibly is quantum state transfer (\cite{QuNoise}})